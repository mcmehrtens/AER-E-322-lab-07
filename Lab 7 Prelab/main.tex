% AER E 361 Mission Report Template
% Spring 2023
% Template created by Yiqi Liang and Professor Matthew Nelson

% Document Configuration DO NOT CHANGE
\documentclass[12 pt]{article}
% --------------------LaTeX Packages---------------------------------
% The following are packages that are used in this report.
% DO NOT CHANGE ANY OF THE FOLLOWING OR YOUR REPORT WILL NOT COMPILE
% -------------------------------------------------------------------

\usepackage{hyperref}
\usepackage{parskip}
\usepackage{titlesec}
\usepackage{titling}
\usepackage{graphicx}
\usepackage{graphviz}
\usepackage[T1]{fontenc}
\usepackage{titlesec, blindtext, color} %for LessIsMore style
\usepackage{tcolorbox} %for references box
\usepackage[hmargin=1in,vmargin=1in]{geometry} % use 1 inch margins
\usepackage{float}
\usepackage{tikz}
\usepackage{svg} % Allows for SVG Vector graphics
\usepackage{textcomp, gensymb} %for degree symbol
\hypersetup{
	colorlinks=true,
	linkcolor=blue,
	urlcolor=cyan,
}
\usepackage{biblatex}
\addbibresource{lab-report-bib.bib}
\usepackage{amsmath}
\usepackage{listings}
\usepackage{multicol}
\usepackage{array}

\usepackage{hologo} %KYR: for \BibTeX
%\usepackage{algpseudocode}
%\usepackage{algorithm}
% This configures items for code listings in the document
\usepackage{xcolor}

\usepackage{fancyhdr} % Headers/Footers
\usepackage{siunitx} % SI units
\usepackage{csquotes} % Display Quote
\usepackage{microtype} % Better line breaks

\definecolor{commentsColor}{rgb}{0.497495, 0.497587, 0.497464}
\definecolor{keywordsColor}{rgb}{0.000000, 0.000000, 0.635294}
\definecolor{stringColor}{rgb}{0.558215, 0.000000, 0.135316}
\definecolor{mygreen}{rgb}{0,0.6,0}
\definecolor{mygray}{rgb}{0.5,0.5,0.5}
\definecolor{mymauve}{rgb}{0.58,0,0.82}

\lstdefinestyle{customc}{
  belowcaptionskip=1\baselineskip,
  breaklines=true,
  frame=L,
  xleftmargin=\parindent,
  language=C,
  showstringspaces=false,
  basicstyle=\footnotesize\ttfamily,
  keywordstyle=\bfseries\color{green!40!black},
  commentstyle=\itshape\color{purple!40!black},
  identifierstyle=\color{blue},
  stringstyle=\color{orange},
 }

 \lstset{ %
  backgroundcolor=\color{white},   % choose the background color; you must add \usepackage{color} or \usepackage{xcolor}
  basicstyle=\footnotesize,        % the size of the fonts that are used for the code
  breakatwhitespace=false,         % sets if automatic breaks should only happen at whitespace
  breaklines=true,                 % sets automatic line breaking
  captionpos=b,                    % sets the caption-position to bottom
  commentstyle=\color{commentsColor}\textit,    % comment style
  deletekeywords={...},            % if you want to delete keywords from the given language
  escapeinside={\%*}{*)},          % if you want to add LaTeX within your code
  extendedchars=true,              % lets you use non-ASCII characters; for 8-bits encodings only, does not work with UTF-8
  frame=tb,	                   	   % adds a frame around the code
  keepspaces=true,                 % keeps spaces in text, useful for keeping indentation of code (possibly needs columns=flexible)
  keywordstyle=\color{keywordsColor}\bfseries,       % keyword style
  language=Python,                 % the language of the code (can be overrided per snippet)
  otherkeywords={*,...},           % if you want to add more keywords to the set
  numbers=left,                    % where to put the line-numbers; possible values are (none, left, right)
  numbersep=8pt,                   % how far the line-numbers are from the code
  numberstyle=\tiny\color{commentsColor}, % the style that is used for the line-numbers
  rulecolor=\color{black},         % if not set, the frame-color may be changed on line-breaks within not-black text (e.g. comments (green here))
  showspaces=false,                % show spaces everywhere adding particular underscores; it overrides 'showstringspaces'
  showstringspaces=false,          % underline spaces within strings only
  showtabs=false,                  % show tabs within strings adding particular underscores
  stepnumber=1,                    % the step between two line-numbers. If it's 1, each line will be numbered
  stringstyle=\color{stringColor}, % string literal style
  tabsize=2,	                   % sets default tabsize to 2 spaces
  title=\lstname,                  % show the filename of files included with \lstinputlisting; also try caption instead of title
  columns=fixed                    % Using fixed column width (for e.g. nice alignment)
}

\lstdefinestyle{customasm}{
  belowcaptionskip=1\baselineskip,
  frame=L,
  xleftmargin=\parindent,
  language=[x86masm]Assembler,
  basicstyle=\footnotesize\ttfamily,
  commentstyle=\itshape\color{purple!40!black},
}

\lstset{escapechar=@,style=customc}

\titlelabel{\thetitle.\quad}

% From here on out you can start editing your document
\newcommand{\subtitle}[1]{%
  \posttitle{%
    \par\end{center}
    \begin{center}\LARGE#1\end{center}
    \vskip0.5em}%
}

\newcommand{\etal}{\textit{et al}., }
\newcommand{\ie}{\textit{i}.\textit{e}., }
\newcommand{\eg}{\textit{e}.\textit{g}., }

% Define the headers and footers
\setlength{\headheight}{70.63135pt}
\geometry{head=70.63135pt, includehead=true, includefoot=true}
\pagestyle{fancy}
\fancyhead{}\fancyfoot{} % clears the headers/footers
\fancyhead[L]{\textbf{AER E 322}}
\fancyhead[C]{\textbf{Aerospace Structures Pre-Laboratory}\\
			  \textbf{Lab 7 Column Buckling}\\
			  Section 4 Group 2\\
			  Matthew Mehrtens\\
			  \today}
\fancyhead[R]{\textbf{Spring 2023}}
\fancyfoot[C]{\thepage}

\begin{document}
\section*{Question 1}
\textit{Review Week \num{9} lecture and corresponding reference book materials (Peery is available online from ISU library).}

\section*{Question 2}
\textit{(\num{15} points) For the end condition of one free and one fixed (lecture note page \num{12}), why is the effective length twice as long as the actual length? Could you come up with a simple explanation? Hint: think of “mirror”...}

The effective length has to be measured from two points on the same vertical axis. The first point is at the free-end, distance $L$ above the ground, and if you imagine the beam is mirrored through the ground, the other point would be at a distance $L$ \textit{beneath} the ground. Therefore, the effective length of the beam is $L+L=2L$.

\section*{Question 3}
\textit{(\num{25} pts) Derive the formulas for critical load $P$ and slenderness ratio $\frac{L}{\rho}$ of a circular rod and a rectangular bar subjected to axial loading, in terms of $\pi$, length $L$, modulus of elasticity $E$ and specimen radius $R$ (for circular rod) or cross-sectional dimensions $B$ and/or $H$ (for rectangular bar).}

We know that
\begin{align}
	P_{cr}&=\frac{\pi^2EI}{L^2}\label{eqn:p_cr}\\
	\frac{L}{\rho}&=L\sqrt{\frac{A}{I}}\label{eqn:slenderness_ratio}
\end{align}
For the circular rod, we know that the moment of inertia is
\begin{align*}
	I&=\frac{\pi}{64}(2R)^4
\end{align*}
Substituting $I$ into Equations \ref{eqn:p_cr} and \ref{eqn:slenderness_ratio}, we derive the following equations for $P_{cr}$ and $\frac{L}{\rho}$:
\begin{align}
	P_{cr}&=\frac{\pi^2E\frac{\pi}{64}(2R)^4}{L^2}=\frac{\pi^3ER^4}{4L^2}\label{eqn:P_cr_rod}\\
	\frac{L}{\rho}&=L\sqrt{\frac{A}{\frac{\pi}{64}(2R)^4}}=\frac{2L}{R^2}\sqrt{\frac{A}{\pi}}\label{eqn:slenderness_ratio_rod}
\end{align}
For the beam, we know the moment of inertia is defined as
\begin{align*}
	I&=\frac{1}{12}BH^3
\end{align*}
Substituting $I$ into Equations \ref{eqn:p_cr} and \ref{eqn:slenderness_ratio}, we derive the following equations for $P_{cr}$ and $\frac{L}{\rho}$:
\begin{align}
	P_{cr}&=\frac{\pi^2E\frac{1}{12}BH^3}{L^2}=\frac{\pi^2EBH^3}{12L^2}\label{eqn:P_cr_beam}\\
	\frac{L}{\rho}&=L\sqrt{\frac{A}{\frac{1}{12}BH^3}}=2L\sqrt{\frac{3A}{BH^3}}\label{eqn:slenderness_ratio_beam}
\end{align}

\section*{Question 4}
\textit{(\num{30} pts) Use the formulas from question three to calculate the $P$s and $\frac{L}{\rho}$s for metal specimens made of stainless 304 annealed cold finish steel (elastic modulus $E=\qty{29000}{ksi}$ and yield strength $\sigma^Y=\qty{35}{ksi}$) and 6061-T6 aluminum ($E=\qty{10000}{ksi}$ and $\sigma^Y=\qty{40}{ksi}$) with the sizes and end conditions given in Table \ref{tbl:question_4_data}. What equivalent lengths will you use for the pivot-pivot and pivot-fixed end conditions? For the \qtyproduct{0.25x1}{in.} aluminum specimen, which dimension do you choose to calculate the slenderness ratio? Hint: see the workout example on pages \numrange{14}{15} in lecture notes. You may want to write yourself a little computer program for these calculations. Tabulate your calculations on the $P$s and $\frac{L}{\rho}$s. Also list the effective lengths you used.}

\begin{table}[!htbp]
\caption{Five column buckling test sets.}
\begin{center}
	\begin{tabular}{|c|c|c|c|c|}
		\hline
		Specimen ID&Material&Cross-Section [\unit{in.}]&Length [\unit{in.}]&End Condition\\
		\hline
		I&aluminum&$\frac{3}{8}$ dia.&\num{30}&both pivot (round)\\
		\hline
		II&aluminum&\numproduct{0.25x1}&\num{30}&both pivot (round)\\
		\hline
		III&steel&$\frac{1}{4}$ dia.&\num{30}&both pivot (round)\\
		\hline
		IV&steel&$\frac{1}{4}$ dia.&\num{24}&both pivot (round)\\
		\hline
		V&steel&$\frac{1}{4}$ dia.&\num{27.5} (\num{30} original)&one pivot, one fixed\\
		\hline
	\end{tabular}
\end{center}
\label{tbl:question_4_data}
\end{table}

I wrote a MATLAB script to solve these problems. Table \ref{tbl:data} shows the resultant $P_cr$ and $\frac{L}{\rho}$ values. The code output and code are shown below.

For configuration \num{2}, we use $\qty{1}{''}$ to be the base and $\qty{0.25}{''}$ to be the height since this results in the smallest moment of inertia, \ie the least resistance to bending. For configurations \numrange{1}{4}, $L_{eff}=L$. For configurations \num{5}, based on the lecture notes, $L_{eff}=0.7L$.

\begin{table}[!htbp]
\caption{Five column buckling test sets.}
\begin{center}
	\begin{tabular}{|c|c|c|c|c|}
		\hline
		Configuration&$P_{cr}$ [\unit{lbf}]&$\frac{L}{\rho}$ []\\
		\hline
		1&106.5&320.0\\
		\hline
		2&142.8&415.7\\
		\hline
		3&60.98&480.0\\
		\hline
		4&95.28&384.0\\
		\hline
		5&124.4&336.0\\
		\hline
	\end{tabular}
\end{center}
\label{tbl:data}
\end{table}

\begin{verbatim}
========== Configuration 1 ==========
P_cr   =  106.452 [lbf]
L/rho  =      320 []
========== Configuration 2 ==========
P_cr   =  142.789 [lbf]
L/rho  =  415.692 []
========== Configuration 3 ==========
P_cr   =  60.9797 [lbf]
L/rho  =      480 []
========== Configuration 4 ==========
P_cr   =  95.2808 [lbf]
L/rho  =      384 []
========== Configuration 5 ==========
P_cr   =  124.448 [lbf]
L/rho  =      336 []
\end{verbatim}
\end{document}
